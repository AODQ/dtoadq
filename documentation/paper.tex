\documentclass{article}
\usepackage{amsmath}
\usepackage{listings}
\usepackage{color}
\usepackage[margin=1.15in]{geometry}

\lstset{frame=tb,
  basicstyle=\footnotesize,
  language=C,
  aboveskip=2mm,
  belowskip=2mm,
  frame=none,
  keepspaces=true,
  showstringspaces=false,
  columns=flexible,
  basicstyle={\small\ttfamily},
  breaklines=false,
  breakatwhitespace=false,
  tabsize=2
}

\lstset{language=C}

\title{DTOADQ}

\begin{document}
  \maketitle
  \section{Introduction}
    I apologize for this being very lengthy and at times mathematically heavy,
    but there is no way of avoiding it.

  \section{SDFs}
  \begin{align*}
    S(P, R) &= P - R^2   & \text{SDF of a sphere Radius  }\\
    S(P, N) &= P \cdot N  & \text{SDF of a plane at Normal}\\
    S(P, H, R) &= <|P_{x, z}| - R, \: P_{y}> - H & \text{SDF of a torus with
                                                  Radius and Hole}\\
  \end{align*}
  \section{OpenCL}
  \section{Light Transport Introduction}
  \section{Bidirectional Path Tracing}
    All the materials in here, about path tracing, have come from Veach's PhD
    thesis. So, in a way, DTOADQ, at this stage, is just an implementation of
    BDPT as described in veach's thesis to the GPU, along with pulling from
    other resources such as Physically Based Render [PBR] 3rd ed. In order
    to understand the mathematics for BDPT, it must be worked up starting from
    light transport. So this section is about transforming the algorithms and
    mathematical models into something easily and efficiently implementable for
    the GPU, even if some of it is a rewrite of the mathematical model so that a
    programmer could understand it. In order to compute a normal of an SDF, you
    must perform a gradient approximation: the central distance onf the SDF at
    P. For example, the difference quotient:
      \[\frac{(f(x+h) - f(x))}{h}\]
    The most obvious way to calculate this is the six-point gradient; where for
    each axis you compute $f(x+h) - f(x-h)$, and normalize the result. This
    returns a vector pointing in the direction where the map's SDF changes the
    most: the normal. A four point gradient is possible; as long as each each
    point of the gradient is multiplied by $\epsilon$ before normalizing. The
    speedup from six to four is worthwhile, as computing each gradient is
    expensive as it involves a sdf map call. There are more utilities with
    gradients not yet currently provided in DTOADQ but are worth mentioning, a
    few are: an additional normal call using an alternative (higher-quality) SDF
    map can be used to compute a 'normal' bumpmap, and analyzing the gradient by
    hand allows a cheap normal computation of primitives such as a spheres.
  
    Rendering equation with solid angle:
  \begin{align}
    L_o(P, \omega_o) = L_o(P, \omega_o) + \int_{\Omega}f_s(P, \omega_i,
    \omega_o) L_i(P, \omega_i) cos\Theta_i d\omega_i
  \end{align}
    To get from solid angle to path, consider:
  \begin{align}
    f_s(v \rightarrow \dot{v} \rightarrow \ddot{v}) &= f_s(P, \omega_i,
                                                      \omega_o)\\
    L_e(v \rightarrow \dot{v}) &= L_e(P, \omega_o)\\
    L_e(v \rightarrow \dot{v}) &= L_e^0(v) L_e^1(v \rightarrow \dot{v})
  \end{align}
    That is, instead of looking at a point $P$ with an incoming and outgoing
    angle $(\omega_i,\:\omega_o)$, look at the path from $P_{-1}$ ($v$) to
    $P_{+1}$ ($\ddot{v}$). More accurately, it's the transport between
    three different surfaces of the scene. In terms of SDFs, the surface
    is every point on an SDF map where $|map| \leq \epsilon$. There's no longer
    any integration over a hemisphere with solid angle $\omega_i$, it's now over
    every single surface in the scene. This will be explained in detail later.
\\
    In the case of $L_e$,
    DTOADQ [at least, for now], only has support for area-light. Thus consider:

  \begin{align}
    L_e(P, \omega_o) &\equiv L_e(P) &\text{[ for area lights ]}\\
    L_e(v \rightarrow \dot{v}) &= L_e^0(v)&
  \end{align}

    The cos term, used for the projected solid angle in irradiance, no longer
    applies when dealing with area. This and $d\omega_i$ now become a
    generalized geometry term, G, that converts a PDF with respect to solid area
    using Jacobian mapping, which involves the inverse squared distance, and the
    cosine angle between the geometric normals at $v$ and $\dot{v}$:

  \begin{align}
    cos\theta_n = v \cdot N(v) \\
    G(v \leftrightarrow \dot{v}) = V(v \rightarrow \dot{v})
        \frac{(cos\theta_n cos\dot{\theta_n})}{|| v - \dot{v}||^2}
  \end{align}

    Where $V$ is the visibility test. Thus the rendering equation with respects
    to area is:
  \begin{align}
    L(v \rightarrow \dot{v}) = L_e(v \rightarrow \dot{v}) \int_ML(v \rightarrow
    \dot{v}) G(v\leftrightarrow \dot{v}) f_s(v\rightarrow \dot{v} \rightarrow
    \ddot{v}) d_A
  \end{align}
    where $A$ is the area on $M$, and $M$ is the union of all scene surfaces.
    This is known as the three-point form or the light transport equation. By
    recursively expanding$^{[1]}$, so that the three point form is integrated
    over every possible set of possible paths, this can be rewritten as the more
    usable format:
  \begin{align} I_j = \int_\Omega f_j(\bar{v}) d\mu(\bar{v}) \end{align}
    Where $\bar{v}$ is a path $z_0\rightarrow z_s \rightarrow y_t \rightarrow
    y_0 (eye \rightarrow light)$, $\Omega$ is the combination of all possible
    paths of any length, and $\mu$ is the area-product measure [the product of
    all the expanded $d_A$]. Special care has to be taken with the definition of
    path itself, as in the source code, there never exists a path $\bar{v}$,
    only eye-path $\bar{z}$ and light-path . Right now, this is analytically
    unsolvable. To compute this, we need to apply monte carlo sampling and limit
    the maximum path length:
  \begin{align}
    I_j \approx \frac{1.0f}{N}\Sigma_0^NF\\
    F = \Sigma_s\Sigma_t\frac{f_j(\bar{v}_{s, t})}{p(\bar{v}_{s, t})}
  \end{align}

    where $s$ is the light-path length, and $t$ is the eye-path length. In
    DTOADQ, they are limited to $s+t \leq 8$[TODO]. Finally, to transform
    $\bar{v}$ into two seperate paths, $\frac{f_j(\bar{v}_{s,
        t})}{p(\bar{v}_{s, t})}$ is split into two with a connection edge $c$
  \begin{align}
    \frac{f_j(\bar{v}_{s, t})}{p(\bar{v}_{s, t})} \equiv
    \frac{f^L(\bar{y})}{p(\bar{y})} c(\bar{y}_s \leftrightarrow \bar{z}_t)
    \frac{f^E(\bar{z})}{p(\bar{z})}  
  \end{align}

    And now
  \begin{align}
    F = \Sigma_s\Sigma_t \alpha^L_s c_{s, t} \alpha^E_t
  \end{align}
  \begin{align}
    \alpha^{L|E}_i =
    \begin{cases}
      \frac{L_e^0(y_0 \rightarrow y_1)}{p_A(y_1)} | \frac{W^0_e(z_0)}{p_A(z_0)},
      &i = 1,\\
      (\frac{f_s(y_{i+1} \rightarrow y_i \rightarrow y_{i-1})}{p_{\sigma}(y_i
      \rightarrow y_{i-1})} | \frac{f_s(z_{i-1} \rightarrow z_{i-2} \rightarrow
      z_{i-3}}{p_\sigma(z_{i-2} \rightarrow z_{i-1})}) \alpha^{L|E}_{i-1},
      &i > 1
    \end{cases}
  \end{align}
    There is no case for $i = 0$ as either, the path doesn't exist, or if one
    does exist, $y_0$ and $z_0$ do not contribute to the image. Veach (and many other
    sources), leave this in, but it's unnecessarily confusing. A special case
    needs to be handled, unfortunately, for $\alpha^E_0$ as it is being
    generated on the fly while $\bar{z}$ is being constructed. The two options
    is to unroll the first iteration of the construction loop, or just allow the
    special case to exist. The latter is the better choice, as all kernels will
    enter and exit the special case at the same time, and the GPU might just
    unroll the loop anyways.
    There may be cases where $s = 0$ ($t = 0$ is not possible without a
    physical camera lens), which is equivalent to just forward path tracing.
    Well, in either case where both paths exist or only the camera path exists,
    the connection strategy below describes how to connect the edges of the two
    paths
  \begin{align}
    c_{s, t} =
    \begin{cases}
      L_e(z_{-2} \rightarrow z_{-1}), &s = 0, t > 0,\\
      f_s(y_{-2} \rightarrow y_{-1} \rightarrow z_{-1}) G(y_{-1} \leftrightarrow
      z_{-1}) f_s(y_{-1} \rightarrow z_{-1} \rightarrow z_{-2}), &s = s, t > 0
    \end{cases}
  \end{align}
    One immediately obvious optimization that could be made with this model is
    in regards to the visibility check in the geometric term, if you were to
    expand $I_j$, and take the V term from outside the G function
    [$G(v \rightarrow
    \dot{v}) V(v \rightarrow \dot{v})$],
    it would be made obvious that all the visibility checks can be
    cancelled out for the entire equation, except for the connection term $c$.
    The expansion is not shown as it is very lengthy, and unnecessary. Paths
    $\bar{y}_{0} \ldots \bar{y}_s$ and $\bar{z}_0 \ldots \bar{z}_t$ being
    visible is intuitively obvious as the paths have been generated using the
    same technique used to check for visibility; raymarching. Specifically, the
    definition of $V$ is
  \begin{align}
    V(v \rightarrow \dot{v}) = ||v - \dot{v}|| \leq
    m(v,\:\overrightarrow{v\:\dot{v}})
  \end{align}
    where $m$ is a march through the SDF map. A visibility check is only
    necessary for the connection edge
    \begin{align*}
      ||v - \dot{v}|| = m(v, \overrightarrow{v, \dot{v}})\\
      \:\:\:\text{for all paths but } y_{-1} \rightarrow z_{-1} 
    \end{align*}
\\
    Moving on, for $\alpha^{L|E}_i$, these are equivalent for both $L$ and $E$,
    the problem is that the $\bar{y}$ path is generated in perspective of the
    light source, so we have to call the BSDF in perspective of the camera; the
    path is flipped. Handling $\bar{y}$ is significantly easier than $\bar{z}$,
    as $\bar{y}$ is precomputed. For $\bar{z}$, the next vertex $z_{+1}$
    ($\omega_o$ in the perspective of SA), is unknown. For example $\alpha^L_1$
    in OpenCL looks like:

    \begin{lstlisting}
      Vertex* V0 = light_path.vertices - 1, * V1 = light_path.vertices;
      float sigma_pdf = Light_PDF(V1.origin, &light);
      float g = Geometric_Term(V0.origin, V1.origin, V0.normal, V1.normal);
      float light_pdf = sigma_pdf * g;
      float Le = light.colour;
      light_contrib[1] = Le / light_pdf;
    \end{lstlisting}

    For $c_{s, t}$, there is no cases for $s = s, t = 0$ as
    the lens does not have a physical existence in the SDF map. $p_A$ is
    the PDF w.r.t. area, for the light path; we only need to take the PDF of the
    solid angle of a sphere
  \begin{align}
    p_A^L(y_0 \rightarrow y_1) = p_{\sigma}^L(y_0 \rightarrow y_1) G(v
    \leftrightarrow \dot{v})\\
    p_{\sigma}^L(y_0 \rightarrow y_1) = \frac{1.0f}{\tau * (1.0f -
    \sqrt{\frac{R^2}{|y_0 - y_1|^2}})}
  \end{align}
    where $R$ is the radius of the sphere, note $y_0$ is a point on the surface
    of the sphere, and not its origin.
    \\$p_A^E(x_0 \rightarrow x_1)$ doesn't exist, as there is no physical lens
    on the scene. [TODO - this might cahnge later]. One of the benefits of this,
    is that there are no special cases for pathes where $t = 0$, which is
    impossible to handle on the GPU as the path could contribute to any
    arbitrary pixel [directly hitting the lens].
  \begin{align}
    P_A^E = 1.0f
  \end{align}
    The $L^0_e$ is contribution from the light source. As of right now, DTOADQ
    only supports area lights, so $L^0_e$ is set a constant for the emitter's
    spectrum value.
    The $W_e^0$ is contribution from the eye source, that is, the lens. Since
    the lens does not exist in the scene, as of now [TODO - this might change
    later], there is no $W_e^0$. While this should be implemented in the future,
    a hidden benefit of the current model is that there are no special cases for
    paths where $t = 0$, which is impossible to handle on the GPU as any such
    path could contribute to any arbitrary pixel [the light path is, after all,
    directly hitting the lens]. So this has now become:
  \begin{align}
    \alpha_i^E = \Sigma_{i\leq 0} 
  \end{align}
  The amount that each path contributes to F differ greatly, so naively
  weighing each contribution uniformly by something like 1.0f, or even, by
  inverse path length, will producy very noisy images, and would not be
  worth the additional computation time that BDPT requires. In order to get good
  samples from the contributions, they must be combined in an optimal manner
  using Multiple Importance Sampling: each contribution is weighted by the PDF
  of the entire path itself:
  \begin{align*}
    F &\equiv \Sigma_s\Sigma_tW_{s, t}C_{s, t}\\
    C_{s, t} &= \alpha^L_s c_{s, t} \alpha^E_t
  \end{align*}
    where $W_{s, t}$ is the MIS weight of path $\bar{y} \leftarrow \bar{z}$, and
    $C_{s, t}$ is the unweighted contribution of the same path. For clarity,
    $C^U_{s, t}$ as described below is the unweighted contribution of the
    algorithm:
  \begin{align}
    C^u_{s, t} = (\alpha^L_s c_{s, t} \alpha^E_t) \equiv
    \frac{f_j(\bar{v})}{p(\bar{v})}
  \end{align}
  thus
  \begin{align}
    F = \Sigma_s\Sigma_tW_{s, t}C^u_{s, t}
  \end{align}
    The formula Veach gives for Multiple Importance Sampling on BDPT is as so:
  \begin{align}
    MIS_{st}(\bar{x}) = \frac{P_s(\bar{x})}{\Sigma_i{P_i(\bar{x})}} 
  \end{align}
    Conceptually, the numerator $P_s(\bar{x})$ is the path density that was
    generated, while the denominator is the path density of other connection
    strategies that could could have, in theory,$^{[1]}$ created the path. The
    straight forward implementation of this has a lot of problems gone
    unresolved in Veach's thesis; PBR 3rd Ed, chapter 16 pg 1014-1016 addresses
    and describe these problems. Primarily, $P_i(\bar{x})$ will overflow
    CLFloat (due to the distance in the Geometric term), and the straight
    forward implementation is very long, hard to debug, and has a poor time
    complexity. But the important part is the end result of their solution:

  \begin{align}
    r_i (\bar{x}) = 
    \begin{cases}
      1.0f, & i = s,\\
      \frac{\overleftarrow{P}(\bar{x_i})}{\overrightarrow{P}(\bar{x_i})} *
      r_{i+1}(\bar{x}), & i < s,\\
      \frac{\overrightarrow{P}(\bar{x_{i-1}})}{\overleftarrow{P}(\bar{x_{i-1}})}
      * r_{i-1}(\bar{x}), & i > s
    \end{cases}
  \end{align}

    However, DTOADQ can simplify a bit further; unlike above, there are two
    paths for which to evaluate. Specifically, in the above term, for where
    $i < s$, we have
    $\frac{\overleftarrow{p}(\bar{x}_i)}{\overrightarrow{p}(\bar{x}_i)}$, 
    which is the light path (i is iterating towards light-length s). The eye
    contribution is inversed from this, (also iterating towards s, hence why
    $\bar{x}_{i-1}$ is used). Another way of saying it, is $\bar{x} = \bar{y} ||
    \bar{z}_{rev}$.
    In DTOADQ's case, the weights are calculated in
    the same direction relative to its originator; in this case the perspective
    of the eye path.
  \begin{align}
    \mathcal{W}_i(\bar{x}) \equiv r_i(\bar{x})\\
    \mathcal{W}_i (\bar{x}) = 
    \begin{cases}
      1.0f, & i = 0,\\
      \frac{\overleftarrow{p_i}(\bar{x_i})}{\overrightarrow{p_i}(\bar{x_i})} *
      \mathcal{W}_{i-1}(\bar{x}), & i > 0
    \end{cases}
  \end{align}
    And then the expanded form of $\mathcal{W}_i(\bar{x})$ looks like

  \begin{align}
    \mathcal{W}_i(\bar{x}) = \Pi_{n=2}^i(\frac{\overleftarrow
    {p_{\sigma}}(x_n)\overleftarrow {G}(x_n)}
    {\overrightarrow{p_{\sigma}}(x_n)\overrightarrow{G}(x_n)})
    * \frac{\overleftarrow{p_A}(x_1)}{\overrightarrow{p_A}(x_1)}
  \end{align}
 
    And now the MIS looks like:
  \begin{align}
    MIS_{s, t}(\bar{y}, \bar{z}) =
    \frac{1.0f}{\mathcal{W}_s(\bar{y}) + \mathcal{W}_t(\bar{z}) + 1.0f}
  \end{align}
    Thus, the final rendering equation for DTOADQ looks like such:

  \begin{align}
    P = \frac{1.0f}{N} \Sigma_{0}^{\infty} \Sigma_{s \le 0} \Sigma_{t \le 0}
    \alpha^L_s c_{s,t} \alpha^E_t MIS_{s, t}(\bar{y}, \bar{z})
  \end{align}


  \begin{align}
    P_i(\bar{x}_{s, t}) = P^L_s p^E_t
  \end{align}

  \begin{align}
    p^{L|E}_i(\bar{x}) =
    \begin{cases}
      1.0f, & i = 0,\\
      P_A(x_i), & i = 1,\\
      P_{\sigma}(x_{i-1} \rightarrow x_i) G(x_{i-1} \leftrightarrow x_i)
      P^{L|E}_{i-1}, & i > 1
    \end{cases}
  \end{align}



  \section{Video/Image Emitter}
  
  \section{Bibliography}
  [1]
  \begin{align*}
    \Sigma_{k=1}^{\infty} \int_{M^{k+1}} L_e(v_0 \rightarrow v_1) G(v_0
    \leftrightarrow v_1)
    \Pi_{i=1}^{k-1} F_s(v_{i-1} \rightarrow v_i \rightarrow v_{i+1}) G(v_i
    \leftrightarrow v_{i+1})\\
    W_e^j(v_{k-1} \rightarrow v_k) dA(v_0) \ldots dA(v_k)
  \end{align*}
\end{document}
